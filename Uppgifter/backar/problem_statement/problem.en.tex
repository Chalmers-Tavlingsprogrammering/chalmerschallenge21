\problemname{Hills in Gothenburg}

As most students are aware, Gothenburg somehow seems to have more roads going uphill than downhill,
regardless of which way you're going. The knowledge of this made Lazy Smurf feel even more hopeless
when his bike broke in the middle of the city, on his way to Smurfette. While the bike could ride
fine downhill, the pedals were broken and he had to lead the bicycle uphill. Now Lazy Smurf needs
your help finding the path around the city to Smurfette which makes him have to lead the bicycle
uphill the least.

As everyone knows, all of Gothenburg can be represented as a graph
\footnote{\url{https://en.wikipedia.org/wiki/Graph_(discrete_mathematics)}}:
a list of $N$ spots in the city, with $M$ direct paths
between the spots (every path leads directly between two spots). Every spot in the city has a known
height $h_i$, and if there is direct path between spot $i$ and $j$, the height difference $h_i-h_j$
is the distance uphill the bicycle needs to be lead. To minimize the amount Lazy Smurf needs to lead
the bicycle, we want to find the way from his location to Smurfette's which minimizes the total
height difference for all steps between the locations, with steps going downhill counting as 0.

\section*{Input}
The first line contains two numbers $N$ and $M$ with a space between them. As previously stated, $N$
is the number of spots in the city, and $M$ is the number of direct paths between them.

On the second row, there are two numbers separated by a space. The first number is the spot where
Lazy Smurf is located, the second is the spot where Smurfette is located.

On the fourth row, there are $N$ numbers separated by spaces. The numbers $h_i$ ($1 \le i \le N$)
represent the height of the $i$th spot.

On each of the following $M$ rows, there will be two numbers $i$ and $j$, meaning that there is a
path between spot $i$ and spot $j$.

\section*{Output}

The program will output one number which is the minimum total uphill distance Lazy Smurf has to
travel to get to Smurfette.

\section*{Notes}

In sample one, the shortest way is going $1 - 2 - 4$, which has a total uphill distance of $1+0=1$.
If you would instead have taken the path $1 - 3 - 4$, the total uphill distance would have been
$0+2=2$.

In sample two, there are a few paths of equal length, $1 - 7 - 6 - 5$ being one of length $0+4+3=7$.
Another path of length $7$ would be $1 - 7 - 4 - 5$. A path of length other than $7$ would be
$1-2-3-4-5$, having a length of $0+3+0+6 = 9$.

\section*{Samples}
