\problemname{CHACTL}
Seasoned competitive programmers are quite familiar with \mathsc{KACTL}, a very good algorithm repository
designed for use in the ICPC. The problem setters of Chalmers Challenge 2021 count themselves to
this group, but have in preparation for this contest noticed two serious flaws with it.

First, it turns out the \mathsc{K} in \mathsc{KACTL} stands for \mathsc{KTH}! This is clearly
unacceptable and needs to be remedied. Second, the $26$-page limit prevents the repository from
achieving true greatness. Hence, we introduce \mathsc{CHACTL}: Chalmers Competition Template Library
with no less than $2^26$ pages. This is plenty space for all distibuted quadratic SIMD, assault and
battery of floating-point numbers and cryptic guides to crossword solving you could ever wish for!

During development, Dragos came up to us and said that our users need a proper navigation system to
handle the scale and intensity of \mathsc{CHACTL}. He recommended buttons that traverse the pages a
power of two at a time. We implemented this, so from page $i$ you can go to page $i + 2^t$ or $i -
2^t$ in a single step for any $t$.

Although there are no pages with negative numbers, or with arbitrarily large numbers, the navigation
system supports jumping to such pages as it might reduce the number of traverals needed.

We want to test this by looking up algorithms on pages $P_0, P_1, \ldots P_{N-1}$, in that order.
\mathsc{CHACTL}'s pages are numbered from $0$ to $2^26 - 1$. Starting from page $0$, what is the
smallest number of traverals required to look up all the algorithms?

\section*{Input}
The first line contains one integer $N$, $1 \le N \le 10^5$. The following $N$ lines each contain
one integer. Line $i$ contains $P_i$, $0 \le P_i \le 2^26$.

\section*{Output}
Output the smallest number of traversals required to look up the algorithms.
